\chapter{结论和展望}{Conclusion and Perspective}

\section{结论}{Conclusion}

为了高效处理海量空间数据,本文以Spark RDD为基础进行空间扩展,构建了
Spatial-Spark并行空间计算框架。以此框架为基础,对海量新浪微博空间数据进行有效
的分析,主要包括新浪微博POI数据和新浪微博用户位置数据分析和挖掘,并对
相关算法进行了并行化改进。

论文主要的工作及其成果总结如下:

(1)对Hadoop MapReduce和Spark并行计算框架进行分析,比较各自的优缺点。重点研究了
HDFS分布式存储策略和RDD的计算特点,分析Spark并行计算框架的优点,主要包括内存
计算带来的性能上优势和丰富的表现算子,为空间数据分析和挖掘提供了可能。

(2)以Spark为基础,对其进行空间方面的拓展,构建了并行空间计算框架Spatial-Spark,主要包括空间
数据分布式读写和空间数据变换,对基本的空间数据类型点、线和面分别拓展成为PointRDD、LineRDD
和PolygonRDD,并根据空间数据分布特征,提供了空间分区功能;
以R树为工具,对每个分区建立的空间索引,以加快空间数据分析速度。在此基础上,提供了
空间数据分析常用的模块,主要包含空间拓扑查询、KNN空间邻近查询和空间连接查询。最后搭建了分布式
Hadoop/Spark计算集群,验证了Spatial-Spark在空间数据处理方面的优势。

(3)分析了新浪微博提供的API接口,使用C\# SDK编写了获取新浪微博POI和全国新浪微博用户位置数据的
应用程序。

(4)研究了空间关联规则算法,重点分析了空间同位规则挖掘挖掘算,并结合Spatial-Spark提出了并行化
空间连接算法优化方案,对全国城市微博POI类别进行空位模式挖掘。首先针对上海、武汉和重庆三市进行
二阶模式挖掘,分析(高等院校、培训机构)模式在不同距离阈值下三市的空间参与度。选择空间距离阈值$d=500\text{\rm{m}}$和空间
参与度阈值$0.6$,对北京市进行同位模式挖掘,发现同位模式阶数越高,空间模式的数量越少,直至六阶,
(KTV,中餐厅,咖啡厅,甜品店,美容美发店,酒吧)为六阶空间模式的实例。由于新浪微博POI人为因素影响
较大,因此这些POI类别集聚的现象更加表达了现实线下人们活动现象,这对商家商业推广提供了很好的建议。

(5)新浪微博用户使用微博位置和用户账号位置形成人口流动,使用Spatial-Spark空间查询构建全国城市
在2016年春节期间人口流动网络图。首先根据GraphX并行计算功能,统计了各个城市人口流入量、流出量和流出
流出比,发现了全国城市人口流动情况的多样性;然后使用PageRank算法,计算出每全国城市在人口流动网络中
的权重并与城市GDP进行关联分析,发现城市权重与城市GDP发展相关联,并确定了各个等级区域人口流动网络的中心;
最后对整个人口网络进行进行社群挖掘,通过改进后模块值法,发现了全国城市人口流动社群,发现城市联系紧密性与省份有关,
地理位置对其影响很大,但也存在突破地理空间位置限制的城市。

\section{展望}{Perspective}

本文以新浪微博为研究内容,使用Spark进行了空间数据分析和挖掘进行了尝试,但是仍需在以下几个方面进行深入研究:

(1)空间连接分区索引使用:空间索引能有效地加快空间查询速度,但在Spatial-Spark上层空间连接模块中尚未有效
地使用已经建立的空间索引的进行加速算法,由于R树构造的特殊性,所以不能通过最小外包矩形进行Join操作。

(2)同位模式算法改进:在新浪微博POI类别同位模式挖掘中,使用全连接算法,虽然通过并行化改进在二阶模式
生成中减低了时间复杂度,但是在生成更高阶的过程中,RDD提供的join操作包含了较多的不满足条件的组合模式,因此
如果改进RDD在join操作的选择是值得研究的方向。

(3)微博数据获取方式改进:由于新浪微博API的限制性,不能获取关于用户的所有微博信息,因此在只能获取用户的
位置同时不能获取用户的历史微博数据,这些导致通过用户微博位置的注册地判断人口流动存在有偏性。如果采用网
络爬虫的方式,突破新浪微博API的限制,获取更多的用户的信息,那么将会发现更加有价值的信息。